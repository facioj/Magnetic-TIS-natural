\documentclass[showpacs, preprintnumbers, pra, superscriptaddress, floatfix, onecolumn, longbibliography]{revtex4-1}
\usepackage{amssymb}
\usepackage{amsmath}
\usepackage{float}
\usepackage{graphicx}
\usepackage{epsfig}
\usepackage[T1]{fontenc}
\usepackage{color}


\usepackage[utf8]{inputenc}
%\topmargin=0.2cm

\newcommand{\beq}{\begin{equation}}
\newcommand{\eeq}{\end{equation}}
\newcommand{\bea}{\begin{eqnarray}}
\newcommand{\eea}{\end{eqnarray}}
\newcommand{\nn}{\nonumber}
\newcommand{\no}{\noindent}
\newcommand{\hs}{\hspace{0.1cm}}
\newcommand{\spz}{\hspace{0.7cm}}
\newcommand{\st}{\stackrel}
\newcommand{\eps}{\epsilon}
\newcommand{\veps}{\varepsilon}
\newcommand{\al}{\alpha}
\newcommand{\s}{\sigma}
\newcommand{\lam}{\lambda}
\newcommand{\om}{\omega}
\newcommand{\iom}{i\omega_n}
\newcommand{\de}{\delta}
\newcommand{\D}{\Delta}
\newcommand{\goto}{\rightarrow}
\newcommand{\lab}{\label}
\newcommand{\be}{\beta}
\newcommand{\zb}{\bar{z}}
\newcommand{\p}{\partial}\newcommand{\vp}{\varphi}
\newcommand{\ra}{\rangle}
\newcommand{\la}{\langle}
\newcommand{\Ga}{\Gamma}
\newcommand{\ga}{\gamma}
\newcommand{\app}{\approx}
\newcommand{\ua}{\uparrow}
\newcommand{\da}{\downarrow}
\newcommand{\Ua}{\Uparrow}
\newcommand{\Da}{\Downarrow}
\newcommand{\dmi}{\frac{1}{2}}
\newcommand{\lra}{\longrightarrow}
\newcommand{\Lra}{\Leftrightarrow}
\newcommand{\tht}{\theta}
\newcommand{\pbf}{}
\newcommand{\SM}{S}
\newcommand{\uul}[1] {\underline{\underline{ #1}}}
\def\mean#1{\left< #1 \right>}

\begin{document}

\title{}


\maketitle

\section{Details of the density-functional (DFT) calculations}

All DFT-based results of this work are obtained with a calculation of a MnBi$_4$Te$_7$ slab made of eight structural blocks (i.e. four septuple layers and four quintuple layers) and of a vacuum of 30 Bohr and based on experimental lattice parameters (Fig. \ref{dft_structure}a)).
We use the LSDA+U method with the generalized grandient approximation as implemented in the FPLO method \cite{}. We fix parameters $U=5.34\,$eV and $J=0$, as in Ref. \cite{}. The spin-orbit interaction is considered in the fully-relativistic four-component formalism. Numerial integrations are performed with a tetrahedron method with a mesh of $12\times12\times1$ subdivisions in the Brillouin zone.

For the simulation of the orbital-projected surface spectral density in Fig. 3 of the main text, we take into account all Bloch states in the slab, weighting their contribution with an exponential decaying factor (Fig. \ref{dft_structure}b). Specifically, denoting the local orbitals as $|\phi^{(z)}_{jm}\rangle$, where $z$ is the distance from the orbital to the surface, for each Bloch state $|\psi_{k_0\nu}\rangle$ of energy $\varepsilon_{k_0\nu}$, we consider as its contribution 
\begin{equation}
A_{jm}(\omega,k)=\frac{1}{\pi}\frac{ \Delta_\varepsilon}{(\omega-\varepsilon_{k\nu})^2 + \Delta^2_\varepsilon}\frac{ \Delta_k}{(k-k_0)^2 + \Delta^2_k }|\langle \phi^{(z)}_{jm} | \psi_{k_0\nu} \rangle|^2 \text{e}^{-z/\lambda},
\end{equation}
where $\lambda=10\,$\AA, $\Delta_\varepsilon=20\,$meV and $\Delta_k=0.005$\AA$^{-1}$. Last, notice that the local orbitals in Fig. 3 of the main text are defined with the cartesian coordinate system shown in Fig. \ref{dft_structure}c).

\begin{figure*}[h!]
 \centering
 \includegraphics[width=14 cm]{structural_data.png}
	\caption{a) Structural model used in the density-functional calculation. b) Illustration of the exponential decaying weight considered for the surface spectral weight simulation. c) Brillouin zone.} 
	\label{dft_structure}
\end{figure*}

\section{Surface density of states}
%\begin{figure*}[h!]
% \centering
% \includegraphics[width=9 cm]{dos.pdf}
%	\caption{	}
%	\label{dft_sl}
%\end{figure*}



\section{Surface state angular momentum reversal within DFT}
Fig. \ref{dft_sl} illustrates the angular momentum reversal between the upper and lower parts of the surface state, as obtained with DFT. 
Fig. \ref{dft_sl}a) presents the band-structure projected on the outmost septuple layer and Fig. \ref{dft_sl}b) shows constant-energy contours. The color-scale reflects the amplitud of the corresponding states on local orbitals of Bi or Te and $J=1/2$, which we found dominate the upper part of the Dirac cone.
Notice that since the experiment of natural dichroism described in the main text is essentially susceptible to the in-plane angular momentum, in this figure the fully relativistic local orbitals considered are defined with the quantization axis along the in-plane cartesian direction $\hat{x}$.
\begin{figure*}[h!]
 \centering
 \includegraphics[width=18 cm]{sl.png}
	\caption{a) MnBi$_4$Te$_7$ band structure projected on the outmost septuple layer (SL-term.) and on the four inner blocks (Bulk). 
	b): Constant energy contours of septuple layer-projected band-structure. In this projection, for simplicity we only consider the Bi or Te states with $J=1/2$, which we found to dominate the upper part of the Dirac cone.
	}
	\label{dft_sl}
\end{figure*}


\bibliography{ref}
\end{document}


